\textbf{Trefethen 6.8}

Let $D_N$ be the usual Chebyshev differentiation matrix. Show that the power $(D_N)^{N+1}$ is identically equal to zero.
Now try it on the computer for $N = 5$ and $N = 20$ and report the computed 2-norms $\Vert (D_5)^{6} \Vert_2$ and 
$\Vert (D_{20})^{21} \Vert_2$. Discuss.

\begin{solution}
  We derived the Chebyshev differentiation matrix by taking the derivative of $N^{th}$ degree Chebyshev interpolating 
  polynomial so that $w_j = p'(x_j)$; as such, each application of the Chebyshev differentiation matrix to
  the interpolating polynomial decreases the degree of the polynomial by one. As a result, the ${(N+1)}^{th}$ 
  application of $D_N$ yields $w_j^{N} = p^{(N+1)}(x_j) = 0$ for each $j$ (since the ${(k + 1)^{th}}$ derivative of any
  $k^{th}$ degree polynomial is zero). This must hold for every Chebyshev interpolation point $x_j$, and so 
  $(D_N)^{N+1}$ must in turn be identically zero.

  Computationally, we find that $\Vert (D_5)^{6} \Vert_2 \approx 1.0 \times 10^{-10}$ and 
  $\Vert (D_{20})^{21} \Vert_2 \approx 6.6 \times 10^{22}$, respectively. Eigenvalues are exponentiated to the same 
  degree as matrices themselves, and for large, ill-conditioned matrices, this causes rapid propagation of machine 
  rounding errors as observed for $(D_{20})^{21}$.
  \ \\
\end{solution}