To simulate experimental observations and generate a reference solution, we utilize a Fourier collocation method along 
with fourth-order Runge-Kutta time stepping. Recall the definition of the Fourier Transform of a differentiable function 
$f$:

$$
\mathcal{F}(f(x)) \equiv \int_{-\infty}^{\infty}f(x)e^{-ikx}dx
$$

where $\hat{f}(k) \equiv \mathcal{F}(f(x))$ is a function in $k$. From this, we also have the inverse Fourier transform:

$$
f(x) = \frac{1}{2\pi}\int_{-\infty}^{\infty} \hat{f}(k)e^{ikx}dk
$$

\noindent Differentiating both sides of the above equation with respect to $k$ yields

$$
\frac{df}{dx} = ik\frac{1}{2\pi}\int_{-\infty}^{\infty}\hat{f}(x)e^{ikx}dx
$$

\noindent and so

$$
\hat{f'}(k) = ik\hat{f}(k).
$$

Taking the inverse Fourier transform of the above equation yields the general form for the derivative of a function with 
respect to its Fourier transform:

$$
f'(x) = \mathcal{F}^{-1}(ik\mathcal{F}(f(x))).
$$

\noindent Extending this to the $n^{th}$ derivative yields:

$$
f^{(n)}(x) = \mathcal{F}^{-1}((ik)^n\mathcal{F}(f(x))).
$$

We apply these results to the SWE system with periodic boundary conditions, which takes the following form:

\begin{align*}
    \begin{bmatrix}
        h\\
        uh
    \end{bmatrix}_t
    + 
    \begin{bmatrix}
        uh\\
        hu^2 + \frac{gh^2}{2}
    \end{bmatrix}_x
    =
    \begin{bmatrix}
        0\\
        -ghB_x
    \end{bmatrix}
\end{align*}    

To compute the spatial derivative, we differentiate in the Fourier domain and apply the inverse Fourier transform 
$\mathcal{F}^{-1}$ to the resulting expression. Upon doing so, our system becomes

\begin{align*}
    \begin{bmatrix}
        h\\
        uh
    \end{bmatrix}_t
    + 
    \mathcal{F}^{-1}\left(ik\mathcal{F}\left(\begin{bmatrix}
        uh\\
        hu^2 + \frac{gh^2}{2}
    \end{bmatrix} \right)\right)
    = 
    \begin{bmatrix}
        0\\
        -ghB_x
    \end{bmatrix}.
\end{align*}

We approximate the time derivative with a fourth order Runge-Kutta time stepping scheme. This system is expressed in 
terms of the time derivatives of $u$ and $uh$; we therefore rewrite the second equation in the system in 
terms of $u$ and $uh$ to find:

$$
\renewcommand*{\arraystretch}{1.5}
\begin{bmatrix}
    h\\
    uh
\end{bmatrix}_t = -\begin{bmatrix}
    \mathcal{F}^{-1}(ik\mathcal{F}(uh))\\
    ghB_x + \mathcal{F}^{-1}(ik\mathcal{F}((uh)^2/h + gh^2/2)) 
\end{bmatrix} = \begin{bmatrix}
    F_1(u,uh)\\
    F_2(u,uh)
\end{bmatrix}.
$$

\noindent Given $h$ at step $n$, our update step to compute $h_{n+1}$ is

\begin{align*}
    k_{1,h} &= dt\frac{L}{2\pi}F_1(h_n, uh_n)\\
    k_{2,h} &= dt\frac{L}{2\pi}F_1(h_n + k_{1,h}/2, uh_n)\\
    k_{3,h} &= dt\frac{L}{2\pi}F_1(h_n + k_{2,h}/2, uh_n)\\
    k_{4,h} &= dt\frac{L}{2\pi}F_1(h_n + k_{3,h}, uh_n)\\
    h_{n+1} &= h_n + \frac{1}{6}(k_{1,h} + 2k_{2,h} + 2k_{3,h} + k_{4,h})
\end{align*}

\noindent As for the quantity $uh$, our update $(uh)_{n+1}$ is given by

\begin{align*}
    k_{1,uh} &= dt\frac{L}{2\pi}F_2(h_{n+1}, uh_n)\\
    k_{2,uh} &= dt\frac{L}{2\pi}F_2(h_{n+1}, uh_n + k_{1,uh}/2)\\
    k_{3,uh} &= dt\frac{L}{2\pi}F_2(h_{n+1}, uh_n + k_{2,uh}/2)\\
    k_{4,uh} &= dt\frac{L}{2\pi}F_2(h_{n+1}, uh_n + k_{3,uh})\\
    uh_{n+1} &= \frac{1}{6}(k_{1,uh} + 2k_{2,uh} + 2k_{3,uh} + k_{4,uh})
\end{align*}

where $dt$ is the time step and $\frac{L}{2\pi}$ the domain scaling factor. To simulate our desired experimental 
measurement data, we solve the SWE system subject to various (sinusoidal) initial conditions and bathymetry functions. 
In the homogeneous scheme, we impose a bathymetry function which is identically zero; for the inhomogeneous system, we
impose a sinusoidal bathymetry. The exact functions used for these computations are described in Section 
\ref{sec:results} and results of these computations for the homogenous system are shown in Figures 
\ref{fig:homogeneous_pseudospectral_swe_height} and \ref{fig:homogeneous_pseudospectral_swe_velocity}; inhomogeneous 
results are shown in Figure \ref{fig:inhomogeneous_pseudospectral_swe_velocity}.