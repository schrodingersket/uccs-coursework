We begin by solving the homogenous system (i.e., we set $\tilde{S}(\textbf{u}) = 0$). The eigenvalues of $A$ are given by

\[\renewcommand\arraystretch{1.5}
\lambda = \begin{pmatrix}
    \lambda_1 \\
    \lambda_2
\end{pmatrix} = \begin{pmatrix}
    v + \sqrt{g h \cos{(\alpha)}} \\
    v - \sqrt{g h \cos{(\alpha)}}
\end{pmatrix}
\]

\noindent with corresponding right eigenvectors\footnote{
    We shall not use these immediately, but will need them later; it is convenient to introduce them alongside the 
    eigenvalues.
}

\[\renewcommand\arraystretch{1.5}
r_1 = \begin{pmatrix}
    \sqrt\frac{h}{g \cos{(\alpha)}} \\
    -1 \\
\end{pmatrix}, \quad r_2 = \begin{pmatrix}
    1 \\
    \sqrt\frac{h}{g \cos{(\alpha)}}
\end{pmatrix}.
\]

\noindent In the $x-t$ plane, our characteristic curves are defined by

\[\renewcommand\arraystretch{1.5}
\frac{d\textbf{x}}{dt} = \textbf{\lambda} (\textbf{u}(x(t), t)) = \begin{pmatrix}
    v + \sqrt{g h \cos{(\alpha)}} \\
    v - \sqrt{g h \cos{(\alpha)}}
\end{pmatrix}
\]

\noindent To determine our Riemann invariants, we seek out (currently unknown) quantitities $W_i$ which are conserved 
(i.e., constant) along some curve $\mathcal{C}$. In particular, we wish to find $\textbf{W}$ along our
characteristic curves, so that:

$$
\frac{d\textbf{W}}{dt} = 0
$$

\noindent along each curve defined in \textbf{x}. We examine the $i^{th}$ component $W_i$ of $\textbf{W}$ which we 
determine by setting the total derivative $\frac{d W_i}{dt} = 0$ along the $i^{th}$ characteristic curve (denoted by 
$x_i(t)$) corresponding to the $i^{th}$ element of \textbf{x}:

\begin{align*}
    0 &= \frac{d W_i}{dt} \\
      &= \frac{\partial W_i}{\partial t} + \frac{\partial W_i}{\partial x_i} \frac{dx_i}{dt} \\
      &= \frac{\partial W_i}{\partial h}\frac{\partial h}{\partial t}
       + \frac{\partial W_i}{\partial v}\frac{\partial v}{\partial t}
       + \left(\frac{\partial W_i}{\partial h}\frac{\partial h}{\partial x}
       + \frac{\partial W_i}{\partial v}\frac{\partial v}{\partial x} \right) \frac{dx_i}{dt} \\
      &= \nabla W_i \cdot \textbf{u}_t + \nabla W_i \cdot \textbf{u}_x \frac{dx_i}{dt}. \\
\end{align*}

\noindent We substitute $\frac{d x_i}{dt} = \lambda_i$ and $\textbf{u}_t = -A(\textbf{u}) \textbf{u}_x$ from the 
previous section to obtain:

$$
-\nabla W_i \left[ A(\textbf{u}) - I \lambda_i \right] \textbf{u}_x = 0.
$$

\noindent and so $-\nabla W_i (\textbf{u})$ must be a left eigenvector of $A(\textbf{u})$ corresponding to $\lambda_i$.
The vector $-\nabla W_i (\textbf{u})$ is therefore a right eigenvector of $A(\textbf{u})^T$, and so the matrix

$$
\begin{pmatrix}
    \vert                   & \cdots & \vert \\
    -\nabla W_1(\textbf{u}) & \cdots & -\nabla W_n(\textbf{u})  \\
    \vert                   & \cdots & \vert
\end{pmatrix} = -\textbf{J}(W)^T
$$

\noindent diagonalizes $A(\textbf{u})^T$, where $\textbf{J}(W)$ is the Jacobian of $W$. We define $L = -\textbf{J}(W)^T$
and write $A(\textbf{u})^T = L \Lambda L^{-1}$. Taking the transpose of both sides yields:

\begin{align*}
    A(\textbf{u}) &= \left( L \Lambda L^{-1} \right)^T   \\
                  &= \left(L^{-1}\right)^T \Lambda^T L^T \\
                  &= L^{-T} \Lambda L^T
\end{align*}

\noindent where $L^{-T}$ denotes the inverse transpose of $L$. We recall from the beginning of this section that the 
right eigenvectors of $A(\textbf{u})$ are given by:

$$
r_1 = \begin{pmatrix}
    \sqrt\frac{h}{g \cos{(\alpha)}} \\
    -1 \\
\end{pmatrix}, \quad r_2 = \begin{pmatrix}
    \sqrt\frac{h}{g \cos{(\alpha)}} \\
    1
\end{pmatrix}.
$$

\noindent and hence the matrix 

$$
R = \begin{pmatrix}
    \vert & \cdots & \vert \\
    r_1   & \cdots & r_n   \\
    \vert & \cdots & \vert
\end{pmatrix}
$$

\ \\
\noindent diagonalizes $A(\textbf{u})$ so that $A(\textbf{u}) = R \Lambda R^{-1}$. Combining this with the above result,
we obtain $L^T = R^{-1}$ and so:

\[\renewcommand\arraystretch{2}
-\textbf{J}(W) = L^T = R^{-1} = \begin{pmatrix}
    \sqrt{\frac{h}{g \cos{(\alpha)}}} & 1 \\
    -1                                & \sqrt{\frac{h}{g \cos{(\alpha)}}}
\end{pmatrix}^{-1} = \begin{pmatrix*}[r]
    \sqrt{\frac{g \cos{(\alpha)}}{4h}} & -\frac{1}{2} \\
    \sqrt{\frac{g \cos{(\alpha)}}{4h}} &  \frac{1}{2} \\
\end{pmatrix*}
\]

\pagebreak

\noindent which at last allows us to explicitly compute the Riemann invariants, which we use to compute upwinded 
numerical flux at cell boundaries in the Discontinuous Galerken method: \footnote{
    We note that multiplying eigenvectors by a scalar value along the way has the effect of scaling the corresponding 
    Riemman invariants. Our Riemann invariants may therefore be equivalently be expressed \linebreak
    $W = (v - 2 \sqrt{gh \cos{\alpha}}, v + 2 \sqrt{gh \cos{\alpha}})^T$.
}

\[\renewcommand\arraystretch{2}
W = \begin{pmatrix}
    \frac{v}{2} - \sqrt{g h \cos{(\alpha)}} \\ 
    \frac{v}{2} + \sqrt{g h \cos{(\alpha)}} 
\end{pmatrix}.
\]
