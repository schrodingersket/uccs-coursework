\begin{mini*}
  {}{z = -5x_1 - 7x_2 - 12x_3 + x_4}{}{}
  \addConstraint{2x_1 + 3x_2 + 2x_3 + x_4}{\le 38}
  \addConstraint{3x_1 + 2x_2 + 4x_3 - x_4}{\le 55}
  \addConstraint{x_1, x_2, x_3, x_4}{\ge 0}.
\end{mini*}

\begin{solution}
  We represent this system in standard form as $Ax = b$, where

  $$
  A = \begin{pmatrix*}[r]
    2 & 3 & 2 &  1 & 1 & 0\\
    3 & 2 & 4 & -1 & 0 & 1 \\
  \end{pmatrix*}, \quad b = \begin{pmatrix*}[r]
    38 \\
    55
  \end{pmatrix*}, \quad x = \begin{pmatrix*}[r]
    x_1 \\
    x_2 \\
    x_3 \\
    x_4 \\
    x_5 \\
    x_6
  \end{pmatrix*}, \quad c = \begin{pmatrix*}[r]
    -5  \\
    -7  \\
    -12 \\
     1  \\
     0  \\ 
     0
  \end{pmatrix*}.
  $$

  With $x_B = \{ x_5, x_6 \}$ as our initial basic feasible solution (corresponding to slack variables), our tableau
  takes the form:

  \[\arraycolsep=8pt\def\arraystretch{2.2}
  \begin{array}{r|rrrrrr|r|r}
    \text{Basic} &  x_1  &  x_2  &  x_3  &  x_4  & x_5 & x_6  & \text{RHS} & \text{(ratio)}  \\ \hline
    -z           &  -5   &  -7   &  -12  &   1   &  0  &  0   &   0        &               \\ \hline
    x_5          &   2   &   3   &   2   &   1   &  1  &  0   &  38        &               \\
    x_6          &   3   &   2   &   4   &  -1   &  0  &  1   &  55        &               
  \end{array}
  \]

  Since $c_N^T$ contains negative entries, this feasible solution is not optimal, and we choose $x_3$ (the most negative
  entry of $c_N^T$) as our entering variable. We compute the ratios of the RHS column to the column of $A$ corresponding
  to $x_3$, select the basic variable with the smallest positive ratio as our exiting variable ($x_6$ in this 
  iteration), and denote the resulting entering and exiting variables in bold in the "Basic" row and column, 
  respectively:
  
  \[\arraycolsep=8pt\def\arraystretch{2.2}
  \begin{array}{r|rrrrrr|r|r}
    \text{Basic} &  x_1  &  x_2  &  \bm{x_3}  &  x_4  & x_5 & x_6  & \text{RHS} & \text{(ratio)}  \\ \hline
    -z           &  -5   &  -7   &  -12       &   1   &  0  &  0   &   0        &                 \\ \hline
    x_5          &   2   &   3   &   2        &   1   &  1  &  0   &  38        & 19              \\
    \bm{x_6}     &   3   &   2   &   4        &  -1   &  0  &  1   &  55        & \sfrac{55}{4}              
  \end{array}
  \]

  We introduce one final bit of notation: $R_{-z}$ denotes the row corresponding to the objective function, and $R_{x_i}$
  denotes the row corresponding to the basic variable $x_i$. We pivot on the $x_3$ column by applying the following 
  sequential row operations:
  
  $$
  -\frac{1}{2} R_{x_6} + R_{x_5}  \longrightarrow 3 R_{x_6} + R_{-z} \longrightarrow \frac{1}{4} R_{x_6}
  $$

  which yields the tableau
  
  \[\arraycolsep=8pt\def\arraystretch{2.2}
  \begin{array}{r|rrrrrr|r|r}
    \text{Basic} &  x_1            &  x_2            &  x_3  &  x_4            &  x_5  &  x_6            &  \text{RHS}     & \text{(ratio)}  \\ \hline
    -z           &   4             &  -1             &  0    &  -2             &   0   &  3              &  165           &                 \\ \hline
    x_5          &   \sfrac{1}{2}  &   2             &  0    &  \sfrac{3}{4}   &   1   &  \sfrac{-1}{2}  &  \sfrac{21}{2} &                 \\
    x_3          &   \sfrac{3}{4}  &   \sfrac{1}{2}  &  1    &  \sfrac{-1}{4}  &   0   &  \sfrac{1}{4}   &  \sfrac{55}{4} &                            
  \end{array}
  \]

  We select $x_4$ as our entering variable, which yields:

  \[\arraycolsep=8pt\def\arraystretch{2.2}
  \begin{array}{r|rrrrrr|r|r}
    \text{Basic} &  x_1            &  x_2            &  x_3  &  \bm{x_4}       &  x_5  &   x_6           & \text{RHS}     & \text{(ratio)}  \\ \hline
    -z           &   4             &  -1             &  0    &  -2             &   0   &   3             &  165           &                 \\ \hline
    \bm{x_5}     &   \sfrac{1}{2}  &   2             &  0    &  \sfrac{3}{2}   &   1   &   \sfrac{-1}{2} &  \sfrac{21}{2} & 7               \\
    x_3          &   \sfrac{3}{4}  &   \sfrac{1}{2}  &  1    &  \sfrac{-1}{4}  &   0   &   \sfrac{1}{4}  &  \sfrac{55}{4} & -55                           
  \end{array}
  \]

  We apply the following row operations to pivot on the $x_4$ column:

  $$
  \frac{1}{4}\left(\frac{2}{3} R_{x_5} \right) + R_{x_3} \longrightarrow 
  2 \left(\frac{2}{3} R_{x_5} \right) + R_{-z} \longrightarrow
  \frac{2}{3} R_{x_5}
  $$

  \pagebreak

  which yields our final tableau:

  \[\arraycolsep=8pt\def\arraystretch{2.2}
  \begin{array}{r|rrrrrr|r|r}
    \text{Basic} &  x_1           &  x_2           &  x_3  &  x_4  &  x_5           &   x_6           & \text{RHS}     & \text{(ratio)}  \\ \hline
    -z           &  \sfrac{14}{3} &  \sfrac{5}{3}  &   0   &  0    &  \sfrac{4}{3}  &   \sfrac{7}{3}  &  179           &                 \\ \hline
    x_4          &  \sfrac{1}{3}  &  \sfrac{4}{3}  &   0   &  1    &  \sfrac{2}{3}  &   \sfrac{-1}{3} &  7             &                 \\
    x_3          &  \sfrac{5}{6}  &  \sfrac{5}{6}  &   1   &  0    &  \sfrac{1}{6}  &   \sfrac{1}{6}  &  \sfrac{31}{2} &                            
  \end{array}
  \]

  Since all elements of $c_N$ are positive, we have arrived at our optimal basic feasible solution, which corresponds to
  the basis $x_B = \{x_3, x_4\}$. This point lies at \linebreak
  $\bar{x} = \left(0, 0, \frac{31}{2}, 7, 0, 0\right)$, and yields an optimal value of $z = -179$.
  \ \\
\end{solution}