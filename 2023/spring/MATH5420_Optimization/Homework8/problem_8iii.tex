How do the results from (i) change if the constraint is replaced by

$$
x^T A x \le 1,
$$

where $A$ is positive definite?

\begin{solution}
    We begin by converting our constraint to "$\ge$" form so that our constraint $g(x)$ is given by:

    $$
    g(x) = 1 - x^T A x \ge 0.
    $$

    The associated Lagrangian is then:

    $$
    \mathcal{L}(x, \lambda) = f(x) - \lambda g(x) = x^T Q x - \lambda (1 - x^T A x).
    $$

    and hence stationary points occur when\footnote{
        Without loss of generality (from lecture), we assume that since $A$ is positive definite, it is also symmetric. 
    }
    
    $$
    \nabla_x \mathcal{L} = Q x + \lambda A x = 0.
    $$

    and hence $Qx = -A (\lambda x)$. Since $A$ is positive definite, it is also invertible, and hence we have

    $$
    -A^{-1} Q x = \lambda x
    $$

    so that our new stationary points are precisely the eigenvectors of $-A^{-1} Q$.
    \ \\
    \vfill
\end{solution}