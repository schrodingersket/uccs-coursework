\textbf{Griva, Nash, Sofer 14.4.1}

Solve the problem

\begin{mini*}
    {}{f(x)  = \frac{1}{2} x_1^2 + x_2^2}{}{}
    \addConstraint{2x_1 + x_2}{\ge 2}
    \addConstraint{x_1 - x_2}{\le 1}
    \addConstraint{x_1}{\ge 0.}
\end{mini*}

\begin{solution}
    For convenience, we enumerate our constraints as follows:

    $$
    g_1(x) = 2 x_1 + x_2 - 2 \ge 0, \quad g_2(x) = x_2 - x_1 + 1 \ge 0, \quad g_3(x) = x_1 \ge 0.
    $$

    The gradient and Hessian of $f$ are given by

    $$
    \nabla f = \begin{pmatrix*}
           x_1 \\
         2 x_2 
    \end{pmatrix*}, \quad \nabla^2 f = \begin{pmatrix*} 
        1 & 0 \\ 
        0 & 2 
    \end{pmatrix*}.
    $$

    Since $\nabla^2 f$ is positive definite everywhere (since its eigenvalues are 1 and 2 for all $x$, which are both 
    positive), it is also positive definite over the null space of the Jacobian of the constraints at any stationary 
    point $x_*$. Hence any stationary point $x_*$ is also a local minimizer. We now proceed to compute stationary points
    by enforcing our first-order necessary condition that $\nabla f = \lambda \nabla g$, which yields

    $$
    \nabla f = \begin{pmatrix*}
           x_1 \\
         2 x_2
    \end{pmatrix*} = \lambda_1 \begin{pmatrix*}
           2 \\
           1
    \end{pmatrix*} + \lambda_2 \begin{pmatrix*}[r]
          -1 \\
           1
    \end{pmatrix*} + \lambda_3 \begin{pmatrix*}
           1 \\
           0
    \end{pmatrix*}.
    $$

    We consider all possible combinations of active constraints and nonzero Lagrange multipliers which satisfy strict 
    complementarity.

    \paragraph{Case I:} $\lambda_1 = g_2 = g_3 = 0.$ \ \\
    If $g_2$ and $g_3$ are active, then $x_1 = 0$ and $x_2 = -1$ and $g_1(x) = 0 - 1 - 2 = -3 < 0$, which violates the
    constraint $g_1(x) \ge 0$. Hence $g_2$ and $g_3$ cannot be simultaneously active.

    \paragraph{Case II:} $\lambda_2 = g_1 = g_3 = 0.$ \ \\
    When $g_1$ and $g_3$ are active, we have $x_1 = 0$ and $x_2 = 2$ (which is feasible, since $g_2(x_1, x_2) = 3 \ge 0$) so
    that 

    $$
    \nabla f = \begin{pmatrix*}
           0 \\
           4
    \end{pmatrix*} = \lambda_1 \begin{pmatrix*}
           2 \\
           1
    \end{pmatrix*} + \lambda_3 \begin{pmatrix*}
           1 \\
           0
    \end{pmatrix*}
    $$
    
    from which we obtain $\lambda_1 = 4$ and $\lambda_2 = -8$. Since $\lambda_2 < 0$, this point is not a local 
    minimizer.

    \paragraph{Case III:} $\lambda_3 = g_1 = g_2 = 0.$ \ \\
    When $g_1$ and $g_2$ are active, we have $x_1 = 1$ and $x_2 = 0$ (which is feasible, since 
    $g_3(x_1, x_2) = 1 \ge 0$) so that 

    $$
    \nabla f = \begin{pmatrix*}
           1 \\
           0
    \end{pmatrix*} = \lambda_1 \begin{pmatrix*}
           2 \\
           1
    \end{pmatrix*} + \lambda_2 \begin{pmatrix*}[r]
          -1 \\
           1
    \end{pmatrix*}.
    $$
    
    from which we obtain $\lambda_1 = \frac{1}{3}$ and $\lambda_2 = -\lambda_1 = -\frac{1}{3}$. Since $\lambda_2 < 0$, this point is not a local 
    minimizer.

    \paragraph{Case IV:} $\lambda_1 = \lambda_2 = g_3 = 0.$ \ \\
    When $g_3$ is active, we have $x_1 = 0$. Our Lagrange multiplier condition then becomes

    $$
    \nabla f = \begin{pmatrix*}
           0 \\
         2 x_2
    \end{pmatrix*} = \lambda_3 \begin{pmatrix*}
           1 \\
           0
    \end{pmatrix*}.
    $$
    
    from which we obtain $\lambda_3 = x_2 = 0$. Hence $x_1 = x_2 = 0$, which is infeasible since 
    $g_1(x_1, x_2) = -2 < 0$.

    \paragraph{Case V:} $\lambda_1 = \lambda_3 = g_2 = 0.$ \ \\
    When $g_2$ is active, we have $x_2 = x_1 - 1$, which yields the Lagrange multiplier condition

    $$
    \nabla f = \begin{pmatrix*}
           x_1 \\
         2 (x_1 - 1)
    \end{pmatrix*} = \lambda_2 \begin{pmatrix*}[r]
          -1 \\
           1
    \end{pmatrix*}.
    $$

    From the first equation, we obtain $x_1 = -\lambda_2$; substituting this expression into the second equation yields
    $\lambda_2 = -\frac{2}{3}$. Since $\lambda_2 < 0$, this point is not a local minimizer.

    \paragraph{Case VI:} $\lambda_2 = \lambda_3 = g_1 = 0.$ \ \\
    When $g_1$ is active, we have $x_2 = 2 - 2 x_1$, which yields the Lagrange multiplier condition

    $$
    \nabla f = \begin{pmatrix*}
           x_1 \\
           4 - 4 x_1
    \end{pmatrix*} = \lambda_1 \begin{pmatrix*}
           2 \\
           1
    \end{pmatrix*}.
    $$

    From the first equation, we obtain $x_1 = 2 \lambda_1$; substituting this expression into the second equation yields
    $\lambda_1 = \frac{4}{9}$ and hence the point $x_* = (x_1, x_2)$ is given by:

    $$
    x_1 = 2 \lambda_1 = \frac{8}{9}, \quad x_2 = 2 - 2 x_1 = \frac{2}{9}
    $$

    Moreover, $g_2(x_1, x_2) = \frac{1}{3} \ge 0$ and $g_3(x_1, x_2) = \frac{8}{9} \ge 0$, so this point is feasible.
    Since $\lambda_1 > 0$, the point $x_*$ satisfies all conditions for a local minimizer.

    \pagebreak
    \paragraph{Case VII:} $\lambda_1 = \lambda_2 = \lambda_3 = 0.$ \ \\
    When all Lagrange multipliers are zero, we have

    $$
    \nabla f = \begin{pmatrix*}
           x_1 \\
         2 x_2
    \end{pmatrix*} = \begin{pmatrix*}
           0 \\
           0
    \end{pmatrix*}.
    $$

    so that $x_1 = x_2 = 0$, which is infeasible (see \textbf{Case IV}).

    \paragraph{Case VIII:} $g_1 = g_2 = g_3 = 0.$ \ \\
    This is simply an extension of \textbf{Case I} and yields the infeasible point $x_1 = 0$, $x_2 = -1$.

    Our only feasible local minimum therefore arises from \textbf{Case VI}, and is given by \linebreak
    $x_* = \left( \frac{8}{9}, \frac{2}{9} \right)$ which yields the optimal value

    $$
    f(x) = \frac{1}{2}x_1^2 + x_2^2 = \frac{4}{9}.
    $$
    \ \\
\end{solution}