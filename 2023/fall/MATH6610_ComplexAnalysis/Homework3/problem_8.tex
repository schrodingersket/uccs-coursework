\textbf{Stein, Shakarchi 3.14}

Prove that all entire functions which are also injective take the form $f(z) = az + b$ with $a, b \in \mathbb{C}$ and 
$a \ne 0$.
\ \\

[\textbf{Hint:} Apply the Casorati-Weierstrass theorem to $f(\sfrac{1}{z})$ and use Theorem 4.4.]

\begin{solution}
  Suppose that $f(z)$ is entire and injective. Since $f$ is entire, there exists a power series expansion about 
  $z_0 = 0$ so that $f = \sum {a_k z^k}$ holds for all $z \in \mathbb{C}$. In particular, since $f$ is holomorphic on 
  $\mathbb{C}$ and injective, the function $g(z) \coloneqq f(\sfrac{1}{z}) = \sum {\frac{a_k}{z^k}}$ is holomorphic on 
  $\mathbb{C} \setminus \{0\}$ and is also injective. Moreover, the injectiveness of $g$ implies that $g$ is 
  non-constant, so that at least one of $a_k$ is non-zero. To show that only finitely many $a_k$ may be non-zero, we 
  suppose for the sake of contradiction that infinitely many $a_k$ are non-zero. The point $z_0 = 0$ is then an 
  essential singularity of $g(z)$, and so by the Casorati-Weierstrass theorem, the image of the punctured unit disc 
  under $g$ is dense in $\mathbb{C}$. Moreover, since $g$ is holomorphic and non-constant in the punctured disc
  $\mathbb{D} \setminus \{0\}$, $g$ is an open function by the open mapping theorem. To obtain our contradiction, we 
  observe that the open ball $B_1(3)$ of radius $1$ centered about $z = 3$ is open and disjoint from the punctured unit 
  disc.  Since $g(\mathbb{D} \setminus \{0\})$ is dense, there exists some point $w_0 \in g(B_1(3))$ which is non-zero 
  (since $f(z)$ is not constant and is injective) whose preimage is in the punctured unit disc which violates the 
  injectiveness of $g$ and yields the desired contradiction. Hence $z_0 = 0$ is a pole singularity of $g$, so that

  $$
  g(z) = a_0 + \sum\limits_{k=0}^{m} \frac{a_k}{z^k}
  $$

  and so $f$ is a polynomial:

  $$
  f(z) = a_0 + \sum\limits_{k=0}^{m} {a_k  z^k}
  $$

  Since $f$ is injective, it may have at most one zero (say, $z_0$), and so $f(z)$ = $a (z - z_0)^m$ for some $m \ge 2$. 
  In particular, if $m > 1$, we have

  $$
  f(e^{\sfrac{2 \pi i}{m}} + z_0) = a \left[e^{\sfrac{2 \pi i}{m}}\right]^m = a e^{2 \pi i} = a = a (1)^m = f(1 + z_0)
  $$

  which contradicts the fact that $f$ is injective. Hence $m = 1$, and with $b \coloneqq -a z_0$ we conclude that
  $f(z) = a z  + b$, as desired.

  \ \\
\end{solution}