Modify \texttt{heat\_CN.m} to solve the heat equation for $-1 \leq x \leq 1$ with step function initial data

\begin{align*}
    u(x,0) = \begin{cases}
          1, \; x < 0 \\
          0, \; x \geq 0.
    \end{cases}
\end{align*} 

With appropriate Dirichlet boundary conditions, the exact solution is

$$
    u(x,t) = \frac{1}{2} \, \text{erfc} \left(x / \sqrt{4 \kappa t}\right),
$$

where erfc is the complementary error function

$$
\text{erfc}(x) = \frac{2}{\sqrt{\pi}} \int_x^\infty e^{-z^2}\,dz.
$$

\begin{enumerate}
    \item
    Test this routine for $m = 39$ and $k = 4h$. Note that there is an initial rapid transient decay of the high wave 
    numbers which is not captured well with this size time step.
    
    \item
    How small do you need to take the time step to get reasonable results? For a suitably small time step, explain why
    you get much better results by using $m = 38$ than $m = 39$.  What is the observed order of accuracy as $k \to 0$ 
    when $k = \alpha h$ with $\alpha$ suitably small and $m$ even?
\end{enumerate}


\begin{solution}\ \\\\
    \hfill\vfill
    \ \\
\end{solution}