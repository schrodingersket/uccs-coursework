Heun's 3rd order method:
\begin{center}
\begin{tabular}{c|ccc}
0 \\
1/3 & 1/3 \\
2/3 &  0  & 2/3\\
\hline \\
    & 1/4 &   0 & 3/4
\end{tabular}
\end{center}

\begin{solution}\ \\\\
    \textbf{(5.35)}

    \begin{flalign*}
    \sum_{j=1}^3 b_{j} &= \frac{1}{4} + 0 + \frac{3}{4} = 1 &\\
    \texttt{i=1: } &\sum_{j=1}^3 a_{1j} = 0 = c_1 &\\
    \texttt{i=2: } &\sum_{j=1}^3 a_{2j} = \frac{1}{3} = c_2 &\\
    \texttt{i=3: } &\sum_{j=1}^3 a_{3j} = 0 + \frac{2}{3} = \frac{2}{3} = c_3 &\\
    \end{flalign*}

    \textbf{(5.38)}

    \begin{flalign*}
    \sum_{j=1}^3 b_{j}c_{j} &= \frac{1}{4} \cdot 0 + 0 \cdot \frac{1}{3} + \frac{3}{4} \cdot \frac{2}{3} 
                             = \frac{1}{2} &
    \end{flalign*}
   
    \textbf{(5.39)}
    
    \begin{flalign*}
    \sum_{j=1}^3 b_{j}c_{j}^2 &= \frac{1}{4} \cdot 0^2 + 0 \cdot \left(\frac{1}{3}\right)^2 + \frac{3}{4} \cdot \left(\frac{2}{3}\right)^2 
                               = \frac{3}{4} \cdot \frac{4}{9} 
                               = \frac{1}{3} &
    \end{flalign*}

    \begin{flalign*}
    \sum_{i=1}^3\sum_{j=1}^3 b_{i}a_{ij}c_{j} &= \frac{1}{4} (0)
                                               + 0 \cdot \frac{1}{3} \cdot 0
                                               + \frac{3}{4} \left(0 + \frac{2}{3} \cdot \frac{1}{3} \right) 
                                               + \frac{3}{4} \cdot \frac{2}{9} 
                                               = \frac{1}{6} &
    \end{flalign*}

    Hence conditions (5.35), (5.38), and (5.39) are satisfied.
    \pagebreak

    To show that the Taylor series expansion of 
    $e^{k \lambda}$ is recovered to order three, we write a single step of this method explicitly and substitute our
    test function $u' = \lambda u = f(u, t)$ along the way:

    \begin{flalign*}
        Y_1 &= u_n + k \sum_{j=1}^3 a_{1j} f(Y_j, t_n + c_j k) &\\
            &= u_n &\\
        Y_2 &= u_n + k \sum_{j=1}^3 a_{2j} f(Y_j, t_n + c_j k) = u_n + k\left( \frac{1}{3} f(Y_1, t_n + k) \right) &\\
            &= u_n + \frac{1}{3} (k \lambda) u_n  &\\
        Y_3 &= u_n + k \sum_{j=1}^3 a_{3j} f(Y_j, t_n + c_j k) = u_n + k\left[ \frac{2}{3} f\left(Y_2, t_n + \frac{1}{3} k\right) \right] &\\
            &= u_n + \frac{2}{3} (k \lambda) u_n + \frac{2}{9} (k \lambda)^2 u_n  &\\
        U^{n+1} &= u_n + k \sum_{j=1}^3 b_j f(Y_j, t_n + c_j k) &\\
                &= u_n + k \left[ \frac{1}{4} f(Y_1, t_n + k) + \frac{3}{4} f\left(Y_3, t_n + \frac{2}{3} k\right) \right] &\\
                &= u_n + \frac{1}{4} k \lambda u_n 
                       + \frac{3}{4} k \lambda \left[ u_n + \frac{2}{3} (k \lambda) u_n + \frac{2}{9} (k \lambda)^2 u_n \right] &\\
                &= u_n + \left( \frac{1}{4} + \frac{3}{4} \right)(k \lambda) u_n
                       + \frac{3}{4} \cdot \frac{2}{3} (k \lambda)^2 u_n
                       + \frac{3}{4} \cdot \frac{2}{9} (k \lambda)^3 u_n &\\
                &= u_n \left[1 + (k \lambda) + \frac{1}{2!} (k \lambda)^2 + \frac{1}{3!} (k \lambda)^3 \right].
    \end{flalign*}
    \ \\
\end{solution}