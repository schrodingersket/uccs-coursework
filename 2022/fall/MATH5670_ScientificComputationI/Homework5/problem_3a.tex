Runge's 3rd order method:
\begin{center}
\begin{tabular}{c|cccc}
0 \\
1/2 & 1/2 \\
1   &   0 & 1 \\
1   &   0 & 0   & 1 \\
\hline \\
    & 1/6 & 2/3 & 0 & 1/6
\end{tabular}
\end{center}

\begin{solution}\ \\\\
    \textbf{(5.35)}

    \begin{flalign*}
    \sum_{j=1}^4 b_{j} &= \frac{1}{6} + \frac{2}{3} + 0 + \frac{1}{6} = 1 &\\
    \texttt{i=1: } &\sum_{j=1}^4 a_{1j} = 0 = c_1 &\\
    \texttt{i=2: } &\sum_{j=1}^4 a_{2j} = \frac{1}{2} = c_2 &\\
    \texttt{i=3: } &\sum_{j=1}^4 a_{3j} = 0 + 1 = 1 = c_3 &\\
    \texttt{i=4: } &\sum_{j=1}^4 a_{4j} = 0 + 0 + 1 = 1 = c_4 &
    \end{flalign*}

    \textbf{(5.38)}

    \begin{flalign*}
    \sum_{j=1}^4 b_{j}c_{j} &= \frac{1}{6} \cdot 0 + \frac{2}{3} \cdot \frac{1}{2} + 0 \cdot 1 + \frac{1}{6} \cdot 1 
                             = \frac{1}{2} &
    \end{flalign*}
   
    \pagebreak
    \textbf{(5.39)}
    
    \begin{flalign*}
    \sum_{j=1}^4 b_{j}c_{j}^2 &= \frac{1}{6} \cdot 0^2 + \frac{2}{3} \cdot \left(\frac{1}{2}\right)^2 + 0 \cdot 1^2 + \frac{1}{6} \cdot 1^2 
                               = \frac{1}{3} &
    \end{flalign*}

    \begin{flalign*}
    \sum_{i=1}^4\sum_{j=1}^4 b_{i}a_{ij}c_{j} &= \frac{1}{6} (0)
                                               + \frac{2}{3} \left(\frac{1}{2} \cdot 0 \right)
                                               + 0 \left(0 + 0 \cdot 1 \cdot \frac{1}{2} \right) 
                                               + \frac{1}{6} ( 1 \cdot 1) 
                                               = \frac{1}{6} &
    \end{flalign*}

    Hence conditions (5.35), (5.38), and (5.39) are satisfied. To show that the Taylor series expansion of 
    $e^{k \lambda}$ is recovered to order three, we apply one step of this
    method to $u' = \lambda u$:

    $$
    $$
    \ \\
\end{solution}