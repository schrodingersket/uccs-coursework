Determine whether each of the following pairs $(X, d)$ define a metric space.

a.  $X = \mathbb{R}, d(x, y) = \sqrt{|x - y|}$ for $x, y \in X$. \ \\

    We first prove the following lemma:

    \emph{Lemma.} If $a, b$ are non-negative numbers in  $\mathbb{R}$, then
    $\sqrt{a} + \sqrt{b} \ge \sqrt{a + b}$.

    \begin{proof}\ \\\\
        Let $a,b$ be non-negative numbers in $\mathbb{R}$. Then the following inequality holds:

        \begin{align*}
            (\sqrt{a} + \sqrt{b})^2 &= (\sqrt{a})^2 + (\sqrt{b})^2 + \sqrt{ab} \\
                                    &= a + b + \sqrt{ab} \\
                                    &\ge a + b.
        \end{align*}

        Taking the square root of both sides\footnote{
            Because $f(x) = \sqrt{x}$ is a monotonically increasing function and because the quantities on both sides of the inequality are 
            positive, the inequality is preserved when taking the square root.
        } yields the desired inequality.
    \end{proof}

    We are now ready to prove the following claim:\ \\

    \emph{Claim.} $d: X \times X \to \mathbb{R}$ defines a metric on $X$.

    \begin{proof}\ \\\\
        Let $x, y, z \in \mathbb{R}$.
       
        Observe that since $|x - y| =  |-(y - x)| = |y - x|$ for all $x, y \in \mathbb{R}$, we have that $d(x, y) = d(y, x)$.

        To show that $d(x, y) = 0$ if and only if $x = y$, we first let $x = y$ and observe that $d(x, y) = \sqrt{|x - y|} = \sqrt{|x - x|} = 0$.
        Now suppose $d(x, y) = 0$. Then $0 = \sqrt{|x - y|}$, and hence $|x - y| = 0$. We therefore have that $x - y = 0$ and $-(x - y) = 0$, and so
        $x = y$, as desired. 

        To prove that the triangle inequality holds for $d$, we observe by the above lemma that:
        
        \begin{align*}
            d(x, z) + d(z, y) &= \sqrt{|x - z|} + \sqrt{|z - y|} \\
                              &\ge \sqrt{|x - z| + |z - y|} \\
                              &\ge \sqrt{|x - z + z - y|} \\
                              &= \sqrt{|x - y|} \\
                              &= d(x, y).
        \end{align*}
    \end{proof}

    \pagebreak

b.  $X = \mathbb{R}, d(x, y) = |x| + |x - y| + |y|$ when $x \neq y$, and 
    $d(x, y) = 0$ when $x = y$ for $x, y \in X$. \ \\

    \emph{Claim.} $d: X \times X \to \mathbb{R}$ defines a metric on $X$.
    \ \\

    \begin{proof}\ \\\\
        Let $x, y, z \in \mathbb{R}$.
        
        We first show that $d$ is transitive:
        \begin{align*}
            d(x, y) &= |x| + |x - y| + |y| \\
                    &= |y| + |-(y - x)| + |x| \\
                    &= |y| + |y - x| + |x| \\
                    &= d(y, x).
        \end{align*}

        To show that $d(x, y) = 0$ if and only if $x = y$, we need only show that $d(x, y) = 0$ implies that
        $x = y$, as the reverse direction is true by the definition of $d$. Suppose that $d(x, y) = 0$ for 
        some $x, y \in \mathbb{R}$. Then $|x| + |x - y| + |y| = 0$ and in particular, $|x - y| = -(|x| + |y|)$.
        Observe that the expression on the left is non-negative and the expression on the right is
        non-positive; hence because the only intersection point of non-negative and non-positive numbers is 0, we
        see that $|x - y| = -(|x| + |y|) = 0$. In particular, $|x - y| = 0$, and by an argument similar to that
        in (a), we have that $x = y$, as desired.

        Lastly, we show that the triangle inequality holds for $d$:

        \begin{align*}
            d(x, z) + d(z, y) &= |x| + |z| + |x - z| + |z| + |y| + |z - y| \\
                              &= |x| + |y| + 2|z| + (|x - z| + |z - y|) \\
                              &\ge |x| + |y| + (|x - z| + |z - y|) \\
                              &\ge |x| + |y| + |x - z + z - y| \\
                              &= |x| + |y| + |x - y| \\
                              &= d(x, y).
        \end{align*}
    \end{proof}

    \pagebreak

c.  $X$ is the space of all Riemann integrable functions on $[a, b]$, and 
    $d(f, g)$ is the function defined by 
    $d(f, g) = \int_a^b{|f(x) - g(x)|dx}$ for $f, g \in X$. \ \\

    \emph{Claim.} $d: X \times X \to \mathbb{R}$ is not a metric on $X$.
    \ \\

    \begin{proof}\ \\\\
        To show that $d$ is not a metric on $X$, we show that $d$ violates the metric axiom that requires
        that $f = g$ whenever $d(f, g) = 0$. In particular, let $g:[a, b] \to \mathbb{R}$ be the function defined
        by $g(x) = 0, x \in [a, b]$, and let $f:[a, b] \to \mathbb{R}$ be defined thus:

        \begin{align*}
            f(x) = \begin{cases}
                0, x \in [a, b) \\
                1, x = b \\
            \end{cases}
        \end{align*}

        Then the function $h \coloneqq f - g = f$ is 
        Then $f$ and $g$ are both non-negative Riemann-integrable functions\footnote{
            In particular, $f = 0$ except at a finite number of points and is thus Riemann integrable.
        } which differ only at a single point; moreover, since $h$ differs from 0 at only a single point
        and is non-zero, we have that:
        
        \begin{align*}
            d(f, g) &= \int_a^b{|f(x) - g(x)|dx} \\
                    &= \int_a^b{|h(x)|dx} \\
                    &= \int_a^b{h(x) dx} \\
                    &= \int_a^b{0 \hspace{1mm} dx} \\
                    &= 0.
        \end{align*}

        Since $f \neq g$ but $d(f, g) = 0$, $d$ is not a metric.
            
    \end{proof}

    \pagebreak

d.  Let $(U, d_U)$ and $(V, d_V)$ be metric spaces, let $X = U \times V$
    be the set of ordered pairs $(u, v)$ with $u \in U$ and $v \in V$, and
    let $d\left((u_1, v_1), (u_2, v_2))\right)$ be the function defined by
    $d\left((u_1, v_1), (u_2, v_2)\right) 
     = \text{max}\left\{d_U(u_1, v_1), d_V(u_2, v_2)\right\}$. \ \\
    
    \emph{Claim.} $d:X \times X \to \mathbb{R}$ defines a metric on $X$.
    \ \\

    \begin{proof}\renewcommand{\qedsymbol}{}\ \\\\
    \end{proof}

    \pagebreak