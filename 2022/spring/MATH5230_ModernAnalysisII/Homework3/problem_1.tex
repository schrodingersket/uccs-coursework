Let $(X, d)$ be a metric space and $A, B \subseteq X$. 

A point $p \in X$ is called an \emph{exterior point} of $A$ provided there exists an open ball of radius $r > 0$ around
$p$ such that $B_r(p) \subseteq X \setminus A$.

A point $p \in X$ is called a \emph{boundary point} of $A$ provided every open ball of radius $r > 0$ contains both
a point in $A$ and a point in $X \setminus A$. Prove the following: \\

a) Prove the following:

i.  $(A \cap B)^{\circ} = A^{\circ} \cap B^{\circ}$, where $A^{\circ}$ and $B^{\circ}$ denote the interior of $A$ and 
    $B$, respectively. \ \\

\begin{proof}\renewcommand{\qedsymbol}{}\ \\\\
    \begin{align*}
    \end{align*}
\end{proof}

\pagebreak

ii.  $(A \cup B)' = A' \cup B'$, where $A'$ and $B'$ denote the set of limit points of $A$ and $B$, respectively.\ \\

\begin{proof}\renewcommand{\qedsymbol}{}\ \\\\
    \begin{align*}
    \end{align*}
\end{proof}

\pagebreak

iii. $A \setminus \text{bd}(A) = A^{\circ}$, where $\text{bd}(A)$ represents the set of boundary points of $A$.  \ \\

\begin{proof}\renewcommand{\qedsymbol}{}\ \\\\
    \begin{align*}
    \end{align*}
\end{proof}

\pagebreak

iv. The set of boundary points $\text{bd}(A)$ is a closed set in $X$.  \ \\
    
\begin{proof}\renewcommand{\qedsymbol}{}\ \\\\
    \begin{align*}
    \end{align*}
\end{proof}

\pagebreak


v. $A^{\circ} \cup \text{bd}(A) = A \cup A'$ \ \\
    
\begin{proof}\renewcommand{\qedsymbol}{}\ \\\\
    \begin{align*}
    \end{align*}
\end{proof}

\pagebreak

b. Prove that if either $A$ is open or it is closed, then $\text{bd}(A)^{\circ} = \emptyset$. Provide a counterexample
   which shows that this assertion fails when $A$ is neither open nor closed. \ \\
    
\begin{proof}\ \\\\
    Suppose that $bd(A)^{\circ}$ is non-empty. We will show that $A$ cannot be open or closed. Since $bd(A)^{\circ}$ is 
    non-empty, there exists some point $x \in bd(A)^{\circ}$, and because $bd(A)^{\circ} \subseteq bd(A)$, we have that
    $x$ is therefore also in the boundary of $A$. Moreover, because the interior of any set is open, there exists an 
    open ball $B_{\epsilon}(x) \subseteq bd(A)$ centered about $x$; since $x \in bd(A)$, $B_{\epsilon}(x)$ also 
    contains a point $a \in A$ and a point $z \in X \setminus A$. If $A$ is open, then there exists an open ball 
    $B_{\delta}(a)$ centered about $a$ which is entirely contained in $A$. However, because $a$ is in the boundary of 
    $A$, every open ball about $a$ must necessarily contain a point in $A$ and a point in $X \setminus A$ , which 
    yields a contradiction. Similarly, if $A$ is closed, then $X \setminus A$ is open, and so there exists an open ball
    $B_{\delta}(z)$ centered about $z$ which is entirely contained in $X \setminus A$, which again contradicts the fact 
    that $z$ is in the boundary of $A$, as before. \\

    By contraposition, we see that if either $A$ is open or it is closed, then $\text{bd}(A)^{\circ} = \emptyset$. \\

    Lastly, observe that the set of rationals $\mathbb{Q} \subset \mathbb{R}$ is neither open nor closed, and that the
    closure of $\mathbb{Q}$ is $\mathbb{R}$. Moreover, $\mathbb{R}$ is open and hence is equal to its interior. Since
    $\mathbb{R}$ is non-empty, we see that $bd(\mathbb{Q})^{\circ} = \mathbb{R} \neq \emptyset$.
    \ \\
\end{proof}

\pagebreak