(\#6 in 2.5) Show that the countable union of countable sets is countable. (Hint: consider a table similar to the one 
used to show that the set of rational numbers is countable).

\begin{proof}\ \\\\
    Let each $S = \bigcup\limits_{i=1}^{\infty}{S_i}$ be a countable collection of countable sets $S_i$, and observe
    that each $S_i$ may be written $S_i = \bigcup\limits_{j=1}^{\infty}{\{s_{ij}\}}, s_{ij} \in S_i$.  Hence $S$ may be
    written $S = \bigcup\limits_{i=1}^{\infty}\bigcup\limits_{j=1}^{\infty}{\{s_{ij}\}}$. Let 
    $f: \mathbb{N} \times \mathbb{N} \to S$ be the function defined by $f((i, j)) = s_{ij}$ for each 
    $i,j \in \mathbb{N}$. Then $f$ is surjection from $\mathbb{N} \times \mathbb{N}$ to $S$. 

    We next show that $\mathbb{N} \times \mathbb{N}$ is countable by constructing a bijection
    $g:\mathbb{N} \to \mathbb{N} \times \mathbb{N} $. We first partition the elements of $\mathbb{N} \times \mathbb{N}$ 
    by the sum of the elements in each pair (i.e., an element $(a, b)$ is similar to $(c, d)$ if and only if 
    $a + b = c + d$) and impose a total ordering of the equivalence classes according to the usual "less than or equal
    to" relation on this sum. Each equivalence class contains a finite number of elements (in particular, for a 
    particular class $[(a, b)]$, there are always precisely $a + b - 1$ elements in the class), and observe that the
    elements within each class may be ordered according to the usual "less than or equal to" relation imposed upon the
    first element in each pair. Then since each equivalence class is a finite set with a total ordering, each such 
    class is well-ordered. We choose the least element of the least equivalence class, and it is to this value that we
    assign $g(1)$. We then assign to $g(2)$ the next least element of the equivalence class, and proceed in this manner
    until all the elements in a particular (finite) equivalence class are exhausted. Since the equivalence classes are
    totally-ordered, we then move to the next least equivalence class and assign each element in that class to a 
    corresponding unique $g(k)$. In this fashion, we see that $g$ uniquely assigns every element of $\mathbb{N}$ 
    to every element in $\mathbb{N} \times \mathbb{N}$, and is therefore a bijection from $\mathbb{N}$ to 
    $\mathbb{N} \times \mathbb{N}$. 
   
    Because $g$ and $f$ are both surjective, the composition $f \circ g:\mathbb{N} \to S$ is a surjection from 
    $\mathbb{N}$ to $S$ and so $S$ is countable.
    \ \\
\end{proof}