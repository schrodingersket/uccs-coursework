Let $(X, d)$ be a metric space and $A, B \subseteq X$. 

A point $p \in X$ is called an \emph{exterior point} of $A$ provided there exists an open ball of radius $r > 0$ around
$p$ such that $B_r(p) \subseteq X \setminus A$.

A point $p \in X$ is called a \emph{boundary point} of $A$ provided every open ball of radius $r > 0$ contains both
a point in $A$ and a point in $X \setminus A$. \\

a) Prove the following:

i.  $(A \cap B)^{\circ} = A^{\circ} \cap B^{\circ}$, where $A^{\circ}$ and $B^{\circ}$ denote the interior of $A$ and 
    $B$, respectively. \ \\

\begin{proof}\ \\\\
    Observe that if $A$ or $B$ are empty, then all of the following hold: $A \cap B = \emptyset$, 
    $(A \cap B)^\circ = \emptyset$, $A^{\circ} = \emptyset$, $B^{\circ} = \emptyset$, and hence 
    $A^{\circ} \cap B^{\circ} = \emptyset$. We may therefore assume that $A$ and $B$ are non-empty sets.

    Let $x \in (A \cap B)^{\circ}$. Then because $(A \cap B)^{\circ}$ is open, there exists an open ball 
    $B_\epsilon(x) \subseteq (A \cap B)^{\circ} \subseteq A \cap B$. Hence $B_\epsilon(x)$ is an open ball which is 
    contained in both $B$ and $A$, and because $A^\circ$ and $B^\circ$ are the largest open sets contained in $A$ and 
    $B$ respectively, we have that $B_\epsilon(x) \subseteq A^\circ$ and $B_\epsilon(x) \subseteq B^\circ$. Thus 
    $B_\epsilon(x) \subseteq A^\circ \cap B^\circ$, and so $x \in A^\circ \cap B^\circ$. Since $x$ was chosen to be
    an arbitrary point in $(A \cap B)^{\circ}$, we have that $(A \cap B)^{\circ} \subseteq A^\circ \cap B^\circ$. \\

    Conversely, suppose $x \in A^\circ \cap B^\circ$. Because $A^\circ$ and $B^\circ$ are open, there exist radii 
    $\alpha > 0$ and $\beta > 0$ such that the open balls $B_\alpha(x) \subseteq A^\circ$ and 
    $B_\beta(x) \subseteq B^\circ$ are fully contained in their respective open sets. Let 
    $\epsilon = \min{(\alpha, \beta)}$, and observe that the open ball $B_\epsilon(x)$ is an open set contained in both
    $A$ and $B$, and hence $B_\epsilon(x) \subseteq (A \cap B)^\circ$. Since $x \in B_\epsilon(x)$ by definition, we
    have that $x \in (A \cap B)^\circ$. Because $x$ was chosen to be an arbitrary point in $A^\circ \cap B^\circ$, we
    have that $A^\circ \cap B^\circ \subseteq (A \cap B)^\circ$ and hence 
    $A^\circ \cap B^\circ = (A \cap B)^\circ$, as desired.
    \ \\
\end{proof}

\pagebreak

ii.  $(A \cup B)' = A' \cup B'$, where $A'$ and $B'$ denote the set of limit points of $A$ and $B$, respectively.\ \\

\begin{proof}\ \\\\
    We first consider the case in which either $A$ or $B$ are empty. Without loss of generality, we may assume that $A$
    is empty. Then $A' = \emptyset$, and so $(A \cup B)' = (B)' = B' = A' \cup B'$. We may therefore assume that $A$ and
    $B$ are non-empty.

    Let $x \in (A \cup B)'$. Then for every $r > 0$, the open ball $B_r(x)$ centered about $x$ contains a least one
    point $p \in A \cup B$ which is not $x$. Let $\{c_n\}_{n=1}^{\infty}$ be the sequence obtained by selecting
    a unique point from $B_{1/n}(x)$ which is not $x$ itself. Then $c_n \in A \cup B$ for every $n \in \mathbb{N}$, and
    so at least one of $A$ or $B$ must be infinite. Without loss of generality, we may assume $A$ to be infinite. We
    may therefore create a new subsequence $\{c_{n_k}\}_{k=1}^{\infty}$ consisting of terms which are in $A$, and 
    observe that every open ball $B_r(x)$ contains at least one element from $\{c_{n_k}\}_{k=1}^{\infty}$, and thus
    contains an element in $A$ which is not $x$. Hence $x$ is a limit point of $A$, and so 
    $x \in A' \subseteq A' \cup B'$. Since $x$ was chosen to be arbitrary, we see that 
    $(A \cup B)' \subseteq A' \cup B'$.

    Conversely, suppose that $x \in A' \cup B'$. Then $x$ is a limit point of at least one of $A$ or $B$; without loss
    of generality, we assume that $x$ is a limit point of $A$. Then every open ball $B_r(x)$ centered about $x$ contains
    a point in $A$ which is not $x$. Moreover, every such point is also in $A \cup B$, and so $x$ is also limit point of 
    $A \cup B$. Since $x$ was chosen to be arbitrary, we have that $A' \cup B' \subseteq (A \cup B)'$ and hence
    $A' \cup B' = (A \cup B)'$, as desired.


    \begin{align*}
    \end{align*}
\end{proof}

\pagebreak

iii. $A \setminus \text{bd}(A) = A^{\circ}$, where $\text{bd}(A)$ represents the set of boundary points of $A$.  \ \\

\begin{proof}\ \\\\
    Let $x \in A \setminus \text{bd}(A)$. Then $x \in A$, and because $x$ is not a boundary point of $A$, there exists 
    an open ball $B_r(x)$ centered about $x$ which contains no point outside of $A$. Hence $B_r(x)$ is an open set 
    contained in $A$ which contains $x$, and so $x$ is an interior point of $A$. Since $x$ was chosen to be arbitrary, 
    we see that $A \setminus \text{bd}(A) \subseteq A^\circ$.

    Now suppose that $x$ is an interior point of $A$. Then $x \in A$, and so it remains to show that $x$ is not in the
    boundary of $A$. For the sake of contradiction, suppose that $x \in \text{bd}(A)$. Then every open ball centered 
    about $x$ contains a point in $A$ and a point not in $A$. But this contradicts the fact that there must exist an 
    open ball centered about $x$ which is entirely contained in $A$ by virtue of the fact that $x$ is an interior point.
    Hence $x$ is not in the boundary of $A$, and so $x \in A \setminus \text{bd}(A)$. Since $x$ was chosen to be 
    arbitrary, we see that $A^\circ \subseteq A \setminus \text{bd}(A)$ and so $A^\circ = A \setminus \text{bd}(A)$, as 
    desired.
    \ \\
\end{proof}

\pagebreak

iv. The set of boundary points $\text{bd}(A)$ is a closed set in $X$.  \ \\
    
\begin{proof}\ \\\\
    Let $x \in X \setminus \text{bd}(A)$. If $x \in A$, then because $x$ is not a boundary point of $A$, there exists 
    some open ball $B_\epsilon(x)$ centered about $x$ which contains no elements in $X \setminus A$; hence 
    $B_\epsilon(x) \subseteq A$, and so $x \in A^\circ$. Similarly, if $x \in X \setminus A$, then there exists some
    open ball $B_\delta(x)$ centered about $x$ which contains no elements in $A$; hence 
    $B_\delta(x) \subseteq X \setminus A$, and so $x \in (X \setminus A)^\circ$. We therefore observe that there exists
    an open ball around every point in $X \setminus \text{bd}(A)$ which is contained completely in the open set 
    $A^\circ \cup (X \setminus A)^\circ$. Hence $X \setminus \text{bd}(A)$ is open, and so $\text{bd}(A)$ is closed.
    \ \\
\end{proof}

\pagebreak


v. $A^{\circ} \cup \text{bd}(A) = A \cup A'$ \ \\
    
\begin{proof}\ \\\\
    Observe that if $A = \emptyset$, then $A^\circ = \text{bd}(A) = A' = \emptyset$ and our claim is trivially true. We 
    may therefore assume that $A$ is non-empty.

    Suppose that $x \in A^\circ \cup \text{bd}(A)$. Then $x$ is in at least one of $A^\circ$ or 
    $\text{bd}(A)$. If $x \in A^\circ$, then $x$ is also in $A$ and hence $x \in A \cup A'$.  Suppose 
    $x \in \text{bd}(A)$. If $x \in A$, then $x \in A \cup A'$ trivially, and so we may assume that 
    $x \in X \setminus A$. Then every open ball centered about $x$ contains a point in $A$ which is different from $x$, 
    and hence $x$ is a limit point of $A$. Hence $x \in A' \subseteq A \cup A'$. Since $x$ was chosen to be arbitrary, 
    we have that $A^\circ \cup \text{bd}(A) \subseteq A \cup A'$.
    \ \\
    Now suppose that $x \in A \cup A'$. Then $x$ is in at least one of $A$ or $A'$. Suppose $x \in A'$ and furthermore
    that $x \in X \setminus A$. Then because $x$ is a limit point, every open ball centered about $x$ contains a point
    in $A$ which is not $x$. Since $x$ itself is not in $A$, every such open ball contains a point in $A$ and a 
    point not in $A$, and so $x$ is in the boundary of $A$. Conversely, if $x$ is
    a limit point of $A$ which is also in $A$, then either every open ball centered about $x$ contains a point in $A$
    and a point in $X \setminus A$ (in which case $x$ is in the boundary of $A$), or there exists an open ball centered 
    about $x$ which contains no points in $X \setminus A$ (in which case $x$ is in the interior of $A$). Thus, 
    $x \in A^\circ \cup \text{bd}(A)$. 
    \ \\\\
    Similarly, if $x \in A \setminus A^\circ$, then no open ball centered about $x$ is
    contained in $A$, and hence every open ball contains a point in $A$ ($x$ itself) and a point in $X \setminus A$ and
    so $x$ is therefore a boundary point of $A$. Hence any point in $A$ is either in the interior of $A$ or is a
    boundary point of $A$, and so $A \subseteq A^\circ \cup \text{bd}(A)$.
    \ \\\\
    Since $x \in A \cup A'$ was chosen to be arbitrary, we have that $A \cup A' \subseteq A^\circ \cup \text{bd}(A)$, 
    and hence $A \cup A' = A^\circ \cup \text{bd}(A)$, as desired.
\end{proof}

\pagebreak

b. Prove that if either $A$ is open or it is closed, then $\text{bd}(A)^{\circ} = \emptyset$. Provide a counterexample
   which shows that this assertion fails when $A$ is neither open nor closed. \ \\
    
\begin{proof}\ \\\\
    Suppose that $\text{bd}(A)^{\circ}$ is non-empty. We will show that $A$ cannot be open or closed. Since 
    $\text{bd}(A)^{\circ}$ is non-empty, there exists some point $x \in \text{bd}(A)^{\circ}$, and because 
    $\text{bd}(A)^{\circ} \subseteq \text{bd}(A)$, we have that $x$ is therefore also in the boundary of $A$. Moreover, 
    because the interior of any set is open, there exists an open ball $B_{\epsilon}(x) \subseteq \text{bd}(A)$ centered
    about $x$; since $x \in \text{bd}(A)$, $B_{\epsilon}(x)$ also contains a point $a \in A$ and a point 
    $z \in X \setminus A$. If $A$ is open, then there exists an open ball $B_{\delta}(a)$ centered about $a$ which is 
    entirely contained in $A$. However, because $a$ is also in the boundary of $A$, every open ball about $a$ must 
    necessarily contain a point in $A$ and a point in $X \setminus A$ , which yields a contradiction. Similarly, if $A$ 
    is closed, then $X \setminus A$ is open, and so there exists an open ball $B_{\delta}(z)$ centered about $z$ which 
    is entirely contained in $X \setminus A$, which again contradicts the fact that $z$ is in the boundary of $A$, as 
    before. \\

    By contraposition, we see that if either $A$ is open or it is closed, then $\text{bd}(A)^{\circ} = \emptyset$. \\

    Lastly, observe that the set of rationals $\mathbb{Q} \subset \mathbb{R}$ is neither open nor closed, and that the
    closure of $\mathbb{Q}$ is $\mathbb{R}$. Moreover, $\mathbb{R}$ is open and hence is equal to its interior. Since
    $\mathbb{R}$ is non-empty, we see that $bd(\mathbb{Q})^{\circ} = \mathbb{R} \neq \emptyset$.
    \ \\
\end{proof}

\pagebreak